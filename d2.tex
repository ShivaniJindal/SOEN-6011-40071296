\documentclass[10pt]{article}

\usepackage{arxiv}
\usepackage[utf8]{inputenc} % allow utf-8 input
\usepackage[T1]{fontenc}    % use 8-bit T1 fonts
\usepackage{hyperref}       % hyperlinks
\usepackage{url}            % simple URL typesetting
\usepackage{booktabs}       % professional-quality tables
\usepackage{amsfonts}       % blackboard math symbols
\usepackage{nicefrac}       % compact symbols for 1/2, etc.
\usepackage{microtype}      % microtypography
\usepackage{lipsum}
\usepackage{graphicx}
\usepackage{blindtext}
\usepackage{adjustbox}
\usepackage{float}
\usepackage[utf8]{inputenc}
\usepackage{enumitem}
\usepackage[noend]{algorithmic}
\usepackage{setspace}
\usepackage{clrscode3e}


\begin{document}

\title{40071296 - Deliverable2}

\LARGE{\author{
	Shivani Jindal \\
	40071296\\
	SOEN 6011\\
	https://github.com/ShivaniJindal/SOEN-6011-D1\\
	Software Engineering Processes\\
	Department of Software Engineering and Computer Science\\
	Concordia University\\
	Montreal\\
https://github.com/ShivaniJindal/SOEN-6011-40071296 }}
\maketitle

\begin{figure} [h]
	\centering
	\includegraphics[height=1in,width=3.8in]{C:/Users/Hp/Desktop/Latex/gamma.jpg}
\end{figure}

\begin{figure}[h]
	\centering
	\includegraphics[height=2.35in,width=4in]{C:/Users/Hp/Desktop/Latex/gf.png}
\end{figure}

\onehalfspacing  
\newpage

\noindent\begin{minipage}{0.3\textwidth}% adapt widths of minipages to your needs
	\includegraphics[width=\linewidth]{C:/Users/Hp/Desktop/Latex/background.jpg}
\end{minipage}%
\hfill%
\begin{minipage}{0.6\textwidth}\raggedleft
	DEBUGGER\\
	Advantages and Disadvantages.
\end{minipage}
\newline
\newline
The Eclipse Platform is designed for building (IDEs) integrated development environments, and arbitrary tools. The Eclipse Platform UI is built around a workbench that includes menu and tool bar actions for running and debugging arbitrary programs, and provides a generic debug perspective better suited to that task. \\
ADVANTAGES:-
\begin{itemize} [noitemsep,topsep=0pt]
	\item When a Java program is launched in debug mode, a debug view shows the processes, threads, and stack frames.
	\item During single stepping, the debugger instructs the editor which source
	code line to highlight. Debug-specific views show the list of breakpoints, the values of variables, and the fields of objects. Breakpoints are represented by a special type of marker.
	\item It don't require several recompiles for what might be a small problem and
	often allow debugger-time altering of the variables, which might be handy as well.
\end{itemize}
	DISADVANTAGES:-
	\begin{itemize} [noitemsep,topsep=0pt]
\item	Sometimes Eclipse can get into a bad state and one have to clear the caches and restart or go through elaborate build-clean-build processes to fix it.
\item	Eclipse run very slowly.
\end{itemize}
 
 \newpage
{\huge QUALITY ATTRIBUTES} \\
\newline
 \textbf{CORRECTNESS:-} \\
 Below are some of the important rules for effective programming which are consequences of the program correctness theory.
 \begin{itemize} [noitemsep,topsep=0pt]
 	\item Defining the problem completely :- \\ I precisely described my function in Deliverable 1 along with its requirements.
 	\item Develop the algorithm and then the program logic :- \\ I mentioned two algorithms named Striling and Lanczos in Deliverable 1 and than used Lanczos for code implementation and it is giving accurate results.
 	\item Developers should pay attention to the clarity and simplicity of your program :- \\ Program is very clean and easy to understand as all sub-functions which are needed to implement Gamma function are coded as separate functions rather than putting everything in main function.
 \end{itemize}
 Types of correctness:-
 \begin{itemize}
 	\item Semantic Correctness:- I tried to use as much as i can the Variable names suitable to task or function performed by the code.
 	\item Syntactic Correctness:- I coded my program adherent to java code rules.	
 \end{itemize}

 \textbf{EFFICIENCY:-}\\
 It  tests the amount of resources required by a program to perform a specific function. \\
 My program is quite efficient as all functional requirements and resources it needs to run are available itself in code as sub-functions.
 \newline
 \textbf{MAINTAINABLE:-}\\
 Maintainability is defined as the degree to which a source code is understood, repaired, or enhanced. \\
 My code is not fully based on main function, it is divide into sub-functions so in case one needs to make changes in code its esay to understand and make without modifying main function. Moreover, i have provided comments inline with code which increase the understand ability of code for readers.
 \newline
\textbf{ ROBUST:-}\\
 Robustness is the ability of a computer system to cope with errors during execution and cope with erroneous input.\\
 My program has support for error handling, exceptions are thrown to handle errors.
 \newline
\textbf{ USABLE:-} \\
 In software engineering, usability is the degree to which a software can be used by specified consumers to achieve quantified objectives with effectiveness, efficiency, and satisfaction in a quantified context of use.\\
 I have used textual interface for running my code. Its very easy, at run time it will show you message stating "Enter  number to compute gamma function" and  gives you result in less than 1 minute. It is very eay to use and very effective and fast.
 

 \newpage
 
 \noindent\begin{minipage}{0.3\textwidth}% adapt widths of minipages to your needs
 	\includegraphics[width=\linewidth]{C:/Users/Hp/Desktop/Latex/cs.jpg}
 \end{minipage}%
 \hfill%
 \begin{minipage}{0.6\textwidth}\raggedleft
 	CHECKSTYLE\\
 	Advantages and Disadvantages.
 \end{minipage}
 \newline
 \newline
 Checkstyle is a development tool to help programmers write Java code that adheres to a coding standard. It automates the process of checking Java code to spare humans of this boring (but important) task. This makes it ideal for projects that want to enforce a coding standard. \\
 \newline
 ADVANTAGES:-
 \begin{itemize} [noitemsep,topsep=0pt]
 	\item Check many aspects of your source code such as it can find class design problems, method design problems. 
 	\item It has the ability to check code layout and formatting issues.
 	\item Checks that the parts of a class or interface declaration appear in the order suggested by the Code Conventions for the Java Programming Language.
 	 \end{itemize}

 DISADVANTAGES:-
 \begin{itemize} [noitemsep,topsep=0pt]
 	\item Java code should be written with ASCII characters only, no UTF-8 support.
 	\item To get valid violations, code have to be compilable, in other case you can get not easy to understand parse errors.
 	\item You cannot determine the full inheritance hierarchy of type.
 	\item You cannot see the content of other files. All files are processed one by one.
 \end{itemize}

\newpage
{\huge MAPPING OF REQUIREMENTS WITH TEST CASES} \\
\newline
\begin{table}[h!]
	\begin{center}
		\label{tab:table1}
		\scalebox{1.5}{
			\begin{tabular}{ l| l  } 
				\textbf{Requirements Identifier} & \textbf{Test Cases} \\
				
				\hline
		     	\newline & \newline\\
				FR1 & testX() \\
				\hline
				\newline & \newline\\
				FR3 & testSin() \\
				\hline
				\newline & \newline\\
				FR4 & testPow(), testLog(), testFracPower()\\
				\hline
				\newline & \newline\\
				FR5 & testFactorial()\\
				\hline
				\newline & \newline\\
				FR6 & testExp()\\
				\hline
				\newline & \newline\\
\end{tabular}}
\end{center}
\end{table}
\newpage
\bibliographystyle{unsrt}  
%\bibliography{references} 
\begin{thebibliography}{1}
	
\bibitem{}
Eclipse Reviews.
\newblock https://www.trustradius.com/products/eclipse/reviews
	
\bibitem{}
Software Engineering.
\newblock https://softwareengineering.stackexchange.com/
questions/168540/what-are-the-advantages-of-using-the-java-debugger-over-println
	
\bibitem{}
Eclipse Platform Technical Overview.
\newblock {\em International Business Machines Corp}, 2006.

\bibitem{}
Checkstyle
\newblock https://checkstyle.sourceforge.io/
\newblock Version: 8.23, Last Published: 2019-07-27 

\bibitem{}
\newblock Wikipedia, W. Usability. [online] En.wikipedia.org.

\bibitem{}
\newblock Wikipedia, W. Correctness. [online] En.wikipedia.org.

\bibitem{}
\newblock Wikipedia, W. Robustness. [online] En.wikipedia.org.

\bibitem{}
 Maintainability
\newblock www.castsoftware.com.

\bibitem{}
Efficiency
\newblock https://economictimes.indiatimes.com/definition/efficiency-testing?from=mdr

\end{thebibliography}
\end{document}
